% Options for packages loaded elsewhere
\PassOptionsToPackage{unicode}{hyperref}
\PassOptionsToPackage{hyphens}{url}
%
\documentclass[
  11pt,
]{article}
\usepackage{lmodern}
\usepackage{amsmath}
\usepackage{ifxetex,ifluatex}
\ifnum 0\ifxetex 1\fi\ifluatex 1\fi=0 % if pdftex
  \usepackage[T1]{fontenc}
  \usepackage[utf8]{inputenc}
  \usepackage{textcomp} % provide euro and other symbols
  \usepackage{amssymb}
\else % if luatex or xetex
  \usepackage{unicode-math}
  \defaultfontfeatures{Scale=MatchLowercase}
  \defaultfontfeatures[\rmfamily]{Ligatures=TeX,Scale=1}
\fi
% Use upquote if available, for straight quotes in verbatim environments
\IfFileExists{upquote.sty}{\usepackage{upquote}}{}
\IfFileExists{microtype.sty}{% use microtype if available
  \usepackage[]{microtype}
  \UseMicrotypeSet[protrusion]{basicmath} % disable protrusion for tt fonts
}{}
\makeatletter
\@ifundefined{KOMAClassName}{% if non-KOMA class
  \IfFileExists{parskip.sty}{%
    \usepackage{parskip}
  }{% else
    \setlength{\parindent}{0pt}
    \setlength{\parskip}{6pt plus 2pt minus 1pt}}
}{% if KOMA class
  \KOMAoptions{parskip=half}}
\makeatother
\usepackage{xcolor}
\IfFileExists{xurl.sty}{\usepackage{xurl}}{} % add URL line breaks if available
\IfFileExists{bookmark.sty}{\usepackage{bookmark}}{\usepackage{hyperref}}
\hypersetup{
  pdftitle={Multi-objective integer programming formulation},
  hidelinks,
  pdfcreator={LaTeX via pandoc}}
\urlstyle{same} % disable monospaced font for URLs
\usepackage{graphicx}
\makeatletter
\def\maxwidth{\ifdim\Gin@nat@width>\linewidth\linewidth\else\Gin@nat@width\fi}
\def\maxheight{\ifdim\Gin@nat@height>\textheight\textheight\else\Gin@nat@height\fi}
\makeatother
% Scale images if necessary, so that they will not overflow the page
% margins by default, and it is still possible to overwrite the defaults
% using explicit options in \includegraphics[width, height, ...]{}
\setkeys{Gin}{width=\maxwidth,height=\maxheight,keepaspectratio}
% Set default figure placement to htbp
\makeatletter
\def\fps@figure{htbp}
\makeatother
\setlength{\emergencystretch}{3em} % prevent overfull lines
\providecommand{\tightlist}{%
  \setlength{\itemsep}{0pt}\setlength{\parskip}{0pt}}
\setcounter{secnumdepth}{-\maxdimen} % remove section numbering
% load packages
\usepackage{amsmath,amsfonts,float,makecell,titletoc,titlesec,tocloft,lineno,booktabs,subfiles,textcomp,tabularx,etoolbox,longtable, adjustbox}
\usepackage[T1]{fontenc}
\usepackage{lmodern}
\usepackage[utf8]{inputenc}
\usepackage[doublespacing]{setspace}
\usepackage[autostyle]{csquotes}

% format captions
\usepackage[labelfont={small,bf}, labelsep=space, font={small}]{caption}

% line numbers
\linenumbers

% allow breaks in equations
\allowdisplaybreaks

% format section headers
\titleformat*{\section}{\large\bfseries}
\titleformat*{\subsection}{\large\bfseries}

% make figures static
\let\origfigure\figure
\let\endorigfigure\endfigure
\renewenvironment{figure}[1][2] {
\expandafter\origfigure\expandafter[H]
} {
\endorigfigure
}

% center tables
\let\Begin\begin

\let\End\end

% make tables in fontnote size font
\AtBeginEnvironment{longtable}{\scriptsize\singlespacing}
\AtBeginEnvironment{multicols}{\scriptsize\singlespacing}
\AtBeginEnvironment{tabular}{\scriptsize\singlespacing}
\AtBeginEnvironment{longtabu}{\scriptsize\singlespacing}

% define struts for tables
\newcommand\T{\rule{0pt}{2.6ex}} % top strut
\newcommand\B{\rule[-1.2ex]{0pt}{0pt}} % bottom strut
\ifluatex
  \usepackage{selnolig}  % disable illegal ligatures
\fi

\title{Multi-objective integer programming formulation}
\author{}
\date{\vspace{-2.5em}16 November 2020}

\begin{document}
\maketitle

We will begin by recalling fundamental concepts in systematic
conservation planning. Conservation features describe the biodiversity
units (e.g.~species, communities, habitat types) that are used to inform
protected area establishment. Planning units describe the candidate
areas for protected area establishment (e.g.~cadastral units). Each
planning unit contains an amount of each feature (e.g.~presence/absence,
number of individuals). A prioritisation describes a candidate set of
planning units selected for protected establishment. Each feature has a
representation target indicating the minimum amount of each feature that
ideally should be held in the prioritisation (e.g.~50 presences, 200
individuals). To minimize risk, we have a set of datasets describing the
relative risk associated with selecting each planning unit for protected
area establishment. Thus we wish to identify a prioritisation that meets
the representation targets for all of the conservation features, with
minimal risk.

We will now express these concepts using mathematical notation. Let
\(I\) denote the set of conservation features (indexed by \(i\)), and
\(J\) denote the set of planning units (indexed by \(j\)). To describe
existing conservation efforts, let \(p_j\) indicate (i.e., using zeros
and ones) if each planning unit \(j \in J\) ia already part of the
global protected area system. To describe the spatial distribution of
the features, let \(A_{ij}\) denote (i.e., using zeros and ones) if each
feature is present or absent from each planning unit. To ensure the
features are adequately represented by the solution, let \(T_i\) denote
the conservation target for each feature \(i \in I\). Next, let \(D\)
denote the set of risk datasets (indexed by \(d\)). To describe the
relative risk associated with each planning unit, let \(R_{dj}\) denote
the risk for planning units \(j \in J\) according to risk datasets
\(d \in D\).

\clearpage

The problem contains the binary decision variables \(x_j\) for planning
units \(j \in J\).

\begin{align*}
x_j &=
\begin{cases}
1, \text{ if $j$ selected for prioritisation}, \tag{eqn 1a} \\
0, \text{ else }
\end{cases} \\
\end{align*}

The reserve selection problem is formulated following:

\begin{align*}
\text{lexmin } & f_1(x), f_2(x), \ldots f_D(x) & \tag{eqn 2a} \\
\text{subject to } & f_d(x) = \sum_{j \in J} R_{dj} X_j & \forall d \in D \tag{eqn 2b} \\
& \sum_{j \in J} A_{ij} \geq T_i & \forall i \in I \tag{eqn 2c} \\
& x_j \geq p_j & \forall j \in J \tag{eqn2d} \\
& x_j \in \{ 0, 1 \} & \forall j \in J \tag{eqn 2e} \\
\end{align*}

The objective function (eqn 2a) is to lexicographically (hierarchically)
minimize multiple functions. Constraints (eqn 2b) define each of these
functions as the total risk encompassed by selected planning units given
each risk dataset. Constraints (eqn 2c) ensure that the representation
targets (\(T_i\)) are met for all features. Constraints (eqn 2d) ensure
that the existing protected areas are selected in the solution. Finally,
constraints (eqns 2e) ensure that the decision variables \(x_j\) contain
zeros or ones.

\end{document}
